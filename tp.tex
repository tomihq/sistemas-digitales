\documentclass[10pt,a4paper]{article}
\usepackage{blindtext}
\usepackage{subcaption}
\usepackage{graphicx}
\usepackage{tikz}
\input{AEDmacros}

\title{Sistemas Digitales}
\author{Tomás Agustín Hernández}
\date{}

\begin{document}
\maketitle


\begin{figure}[b]
    \centering
    \begin{tikzpicture}[remember picture,overlay]
        \node[anchor=south east, inner sep=0pt, xshift=-1cm, yshift=2cm] at (current page.south east) {
            \begin{minipage}[b]{0.5\textwidth}
                \includegraphics[width=\linewidth]{logo_uba.jpg}
                \label{fig:bottom}
            \end{minipage}
        };
    \end{tikzpicture}
\end{figure}

\newpage

\section*{Introducción}
\subsection*{Magnitud}
Llamamos magnitud al tamaño de algo, dicho en una medida específica. Es representada a través de un sistema.
Este sistema debe cumplir 3 conceptos fundamentales: 
\begin{itemize}
    \item Finito: Debe haber una cantidad finita de elementos.
    \item Composicional: El conjunto de elementos atómicos deben ser fáciles de implementar y componer.
    \item Posicional: La posición de cada dígito determina en qué proporción modifica su valor a la magnitud total del número.
  \end{itemize}
Algunos de los sistemas de representación más utilizados son: binario, octal, decimal y hexadecimal.


\end{document}
