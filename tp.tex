\documentclass[10pt,a4paper]{article}
\usepackage{blindtext}
\usepackage{subcaption}
\usepackage{graphicx}
\usepackage{tikz}
\usepackage{amssymb}
\usepackage{caption}
\usepackage{amsmath}
\input{AEDmacros}

\title{Sistemas Digitales}
\author{Tomás Agustín Hernández}
\date{}

\begin{document}
\maketitle


\begin{figure}[b]
    \centering
    \begin{tikzpicture}[remember picture,overlay]
        \node[anchor=south east, inner sep=0pt, xshift=-1cm, yshift=2cm] at (current page.south east) {
            \begin{minipage}[b]{0.5\textwidth}
                \includegraphics[width=\linewidth]{logo_uba.jpg}
                \label{fig:bottom}
            \end{minipage}
        };
    \end{tikzpicture}
\end{figure}

\newpage

\section{Introducción a los sistemas de representación}
\subsection*{Magnitud}
Llamamos magnitud al tamaño de algo, dicho en una medida específica. 
Es representada a través de un sistema que cumple 3 conceptos fundamentales: 
\begin{itemize}
    \item Finito: Debe haber una cantidad finita de elementos.
    \item Composicional: El conjunto de elementos atómicos deben ser fáciles de implementar y componer.
    \item Posicional: La posición de cada dígito determina en qué proporción modifica su valor a la magnitud total del número.
  \end{itemize}
Algunos de los sistemas de representación más utilizados son: binario, octal, decimal y hexadecimal.

\subsection*{Bases}
Una base nos indica la cantidad de símbolos que podemos utilizar para poder representar determinada magnitud.
\begin{table}[h!]
    \centering
    \begin{tabular}{|c | c|}
    \hline
    \textbf{Base} & \textbf{Símbolos disponibles}  \\[0.1cm]
    \hline\hline
    2 (binario) & 0, 1 \\
    8 (octal) & 0, 1, 2, 3, 4, 5, 6, 7 \\
    10 (decimal) & 0, 1, 2, 3, 4, 5, 6, 7, 8, 9 \\
    16 (hexadecimal) & 0, 1, 2, 3, 4, 5, 6, 7, 8, 9, A, B, C, D, E, F \\
    \hline
    \end{tabular}
    \caption{Bases más utilizadas}
    \label{tab:bases}
\end{table}

La tabla anterior representa los símbolos disponibles para las bases 2, 8, 10 y 16.\\


Consideremos por un momento que estamos en binario; ¿sería correcto que 1 + 1 = 2? ¡No! Porque 2 no es un símbolo válido en base 2.\\

Para indicar la base en la que está escrito un número, se coloca la base entre paréntesis en la esquina inferior derecha.\\

\(1024_{(10)}\): 1024 representado en base 10 (decimal)

    
\subsection*{Digítos/Bits}
Sea \( n \in \mathbb{Z} \), cuando decimos que tenemos n bits es lo mismo que decir que tenemos n dígitos.
\\
\begin{itemize}
\item 0001: Representa el número 1 en binario, en 4 bits/dígitos.
\item 0010: Representa el número 2 en binario, en 4 bits/dígitos.
\end{itemize}

\subsection*{Teorema de división}
Es una manera de poder realizar un cambio de base de un número decimal a otra base. La representación en la otra base es el resto visto desde abajo hacia arriba.

\[a = k \ast d + r \ con \ 0 \le r < \longitud{d}\]

Pasaje del número \(1024_{(10)}\) a \(1024_{(2)}\)

\[1024 = 512 \ast 2 + 0\]





\end{document}